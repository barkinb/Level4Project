    
\documentclass[11pt]{article}
\usepackage{times}
    \usepackage{fullpage}
    
    \title{Probabilistic nomograms}
    \author{Barkin Bryce - 2452842}

    \begin{document}
    \maketitle
    
    
     

\section{Status report}

\subsection{Proposal}\label{proposal}

\subsubsection{Motivation}\label{motivation} 



Nomograms are instruments designed to find the missing value in a mathematical calculation without a calculator by extending a ruler across multiple axes to find the interception on the axis of the missing value. They were very popular before the invention of modern computers. In most nomograms, the value of one or more variables is represented by a fixed value. 
Some tools can create nomograms, such as PyNomo, and other tools can visualise probability distributions; however, while nomograms faded in popularity due to the development of scientific calculators and interactive probability distributions became more popular and easier to access,a tool that combines the ease of use of a nomogram and the robust information that a probability visualiser can produce could not be found. Such a tool can allow a user to easily calculate a desired value without using a calculator, trust the results, and better understand the relationship between the probability of having a specific value and its impact on the resulting calculation. This can be useful in a financial application where a mortgage broker can show their client the effect of rising interest rates on their mortgage rates, where the interest rate can be based on the monetary interest rate probabilities of the central bank, such as the CME FedWatch.           


\subsubsection{Aims}\label{aims}



This project aims to create a tool on top of an existing nomogram that will allow a user to represent the value of a missing variable(s) using a probability distribution. This tool will enable users to select the nomogram they want to work on, select the variable to apply the probability distribution and figure out the missing value using rays projected on the nomogram. The tool's success can be measured by giving a user a nomogram and probability function and calculating the success rate. It can also be measured qualitatively by user surveys to determine the ease of use. 

\subsection{Progress}\label{progress}

\begin{itemize}
    \item Background research on the history, uses, types and construction of nomogram has been made, and considered complete. Research on probabilistic distributions has also been made.
    \item Language and GUI framework chosen (Python and Tkinter, respectively). 
    \item Framework for statistical calculations has also been chosen, including but not limited to Numpy, scipy, and PyMC. 
    \item Initial version of the GUI has been developed based on a wireframe. 
    
    \item The tool allows users to select an existing nomogram and place it on a canvas.
    \item They can choose the coordinates of an axis to create a computer-generated overlay, for example, a line or a curve, using Bezier curves. 
    
    \item There is a good idea of where to take the project next.  
    
\end{itemize}

\subsection{Problems and risks}\label{problems-and-risks}

\subsubsection{Problems}\label{problems}


\begin{itemize}
    \item The main problem faced during the development of the program was extracting the major axes from an existing nomogram. Using computer vision methods on their own was difficult due to the large variability in axis tick sizes, fonts, shapes, etc, in already created nomograms. Therefore, a manual approach had to be developed where a user selected the coordinates manually to create control points for the Bezier curves to generate a computer-drawn nomogram to visualise the probabilistic distributions and calculations. 
    
    
\end{itemize}

\subsubsection{Risks}\label{risks}


\begin{itemize}
    \item An issue likely to arise is a discrepancy between true and user-selected axes due to inaccuracy caused by manual coordinate selection. This can lead to a chain of inaccuracies in calculating the desired value.  \textbf{Mitigation: }computer-guided selection by limiting the coordinate selection to areas that can be detected by computer vision.  
    
\end{itemize}

\subsection{Plan}\label{plan}

\emph{{[}Time plan, in roughly weekly to monthly blocks, up until
submission week{]}}

    
    \begin{itemize}
        \item Week 1 - \textbf{Deliverable:} Draft submission of Introduction and Background 
        \item Week 2-3 - Work on implementing as many probability distributions as possible 
        \textbf{Deliverable:} Draft submission of Requirements Analysis
        \item Week 4-5 - Constructing interactive visualisations through rays
        \textbf{Deliverable:} Draft submission of Design
        \item Week 6-7 - Evaluation and analysis through user testing 
        \textbf{Deliverable:} Mostly finished product
        \item Week 8-9 - Final changes 
        \textbf{Deliverable:} Draft submission of Evaluation
        \item Week 10-11 - Final write-up and improvements
        \textbf{Deliverable:} Conclusion and Final submission

        
    \end{itemize}
    
    \end{document}
